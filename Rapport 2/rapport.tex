\documentclass[parskip=half]{scrartcl}
	
\usepackage[english]{babel}
\usepackage[latin1]{inputenc}
\usepackage{enumerate} % til \begin{enumerate}[(i)]
\usepackage{enumitem} % til at styre lister globalt
\usepackage{latexsym} % symboler
\usepackage{amsthm} % til thm osv
\usepackage{amssymb} % flere symbloer
\usepackage{bm} % bold math symbols
\usepackage{amsmath} % til pmatrix
\usepackage[hyphens]{url} %til \url med bindestreger
%\PassOptionsToPackage{hyphens}{url}\usepackage{hyperref}
\usepackage[pdftex]{graphicx}	
\usepackage{scrlayer-scrpage} % page setup

\usepackage{framed}
\usepackage[cache=false]{minted} % for source code
\usepackage{xcolor}
%set page header
\chead{FYS-STK4155 --- project 1}


% List will number things using (a), (b), ...
\setenumerate[1]{label={(\alph*)}} % Global setting

% Title setup
\title{Report for project 1}
\date{\today}
\author{Adam P W S{\o}rensen}

%%%%%%%%%%%%
% make chi subscripts low enough
% from http://tex.stackexchange.com/questions/191551/greek-chis-subscript-expressions-how-to-make-it-smaller-or-offset-lower
\let\latexchi\chi
\makeatletter
\renewcommand\chi{\@ifnextchar_\sub@chi\latexchi}
\newcommand{\sub@chi}[2]{% #1 is _, #2 is the subscript
  \@ifnextchar^{\subsup@chi{#2}}{\latexchi^{}_{#2}}%
}
\newcommand{\subsup@chi}[3]{% #1 is the subscript, #2 is ^, #3 is the superscript
  \latexchi_{#1}^{#3}%
}
\makeatother

\newcommand{\setof}[2]{\left\{ #1 \; \middle\vert \; #2 \right\}}

\DeclareMathOperator{\cspan}{\overline{span}}

% Theorem opsætning
\newtheorem{theorem}{Theorem}[section]
\newtheorem*{theorem*}{Theorem}
\newtheorem{lemma}[theorem]{Lemma}
\newtheorem{corollary}[theorem]{Corollary}
\newtheorem{proposition}[theorem]{Proposition}
\newtheorem{example}[theorem]{Example}
\newtheorem{conjecture}[theorem]{Conjecture}
\theoremstyle{definition}
\newtheorem{definition}[theorem]{Definition}
\newtheorem{assumption}[theorem]{Standing Assumption}
\newtheorem*{assumption*}{Standing Assumption}
\theoremstyle{remark}
\newtheorem{remark}[theorem]{Remark}
\newtheorem{notation}[theorem]{Notation}

%%%%%%%%%%%%
% notation short cuts
\newcommand{\vect}[1]{{\bm{#1}}}
\newcommand{\funcname}[1]{{\color{blue}{\texttt{#1}}}}
\newcommand{\varname}[1]{\texttt{#1}}
%%%%%%%%%%%%
% bb letters
\newcommand{\C}{\mathbb{C}}
\newcommand{\E}{\mathbb{E}}
\newcommand{\N}{\mathbb{N}}
\newcommand{\Q}{\mathbb{Q}}
\newcommand{\R}{\mathbb{R}}
\newcommand{\Z}{\mathbb{Z}}
% cal letters
\newcommand{\A}{\mathcal{A}}
\newcommand{\B}{\mathcal{B}}
\newcommand{\D}{\mathcal{D}}
\newcommand{\cL}{\mathcal{L}}
\newcommand{\M}{\mathcal{M}}
\newcommand{\cZ}{\mathcal{Z}}
% frak letters
\newcommand{\fA}{\mathfrak{A}}


%%%%%%%%%%%%%%
\begin{document}
%%%%%%%%%%%%%%

\maketitle

\begin{abstract}
What we did in project 2
\end{abstract}


\section{Introduction}


\section{Methods} \label{sec:methods}
 
\subsection{Linear regression} \label{sec:linear} 

The three linear regression methods I have used - ordinary least squares regression, ridge regression, and lasso regression - are discussed in detail in the report for project 1. 
So instead of describe how they works, let us zoom in on an import distinction between the way linear regression methods and neural networks were used for regression in this project.  

Suppose we are preforming a biology experiment.
For each test person we record their height and weight and perform a measurement of their cholesterol level. 
Now the question is if height and weight somehow can predict cholesterol level. 
Using linear regression we now have to decide on a model.
Do we think cholesterol can be predicted simply by height and weight, or do we need more features like powers of height and weight and products of these powers. 
Perhaps we could even consider using Body Mass Index (BMI) (it appears that there is in fact some correlation between BMI and cholesterol levels \cite{somepaper}).
Each of these choices will lead to different models we have to choose between.
But in each case the coefficients will give an indication of how much each future matters.
This is a real strength of linear regression methods. 

In contrast to this, as we will discuss below, when we do neural networks, we aim to find a network configuration that can be trained to give good predictions when we feed in height and weight. 
And it can be quite daunting to review the networks weights and biases to deduce how each feature contributes.     

\subsection{Logistic regression} \label{sec:logistic} 





\subsection{Neural networks} \label{sec:neuralnet} 
 
\section{1 dimensional Ising data (parts b and d)} \label{sec:1dising}

\section{2 dimensional Ising data (parts c and e)} \label{sec:2dising}

\section{Conclusions (part f)} \label{sec:conclusion}

From the modelling of the Franke Function and the terrain date, we see that the linear regression methods acts as the theory predicts.
The Ridge and Lasso regressions are more stable than the Ordinary Least Squares regression and Lasso is much more likely to set a given $\beta_i$ to $0$. 
None of the linear regression methods are particularly well suited for these particular problems. 
Hence other methods are needed if we want useful predictions. 

%%%%%%%%%%%%%%
%Bibliography
\bibliographystyle{apalike}
\bibliography{refs}	% expects file "refs.bib"
%%%%%%%%%%%%%%


\end{document}